\documentclass[11pt]{article}

    \usepackage[breakable]{tcolorbox}
    \usepackage{parskip} % Stop auto-indenting (to mimic markdown behaviour)
    
    \usepackage{iftex}
    \ifPDFTeX
    	\usepackage[T1]{fontenc}
    	\usepackage{mathpazo}
    \else
    	\usepackage{fontspec}
    \fi

    % Basic figure setup, for now with no caption control since it's done
    % automatically by Pandoc (which extracts ![](path) syntax from Markdown).
    \usepackage{graphicx}
    % Maintain compatibility with old templates. Remove in nbconvert 6.0
    \let\Oldincludegraphics\includegraphics
    % Ensure that by default, figures have no caption (until we provide a
    % proper Figure object with a Caption API and a way to capture that
    % in the conversion process - todo).
    \usepackage{caption}
    \DeclareCaptionFormat{nocaption}{}
    \captionsetup{format=nocaption,aboveskip=0pt,belowskip=0pt}

    \usepackage[Export]{adjustbox} % Used to constrain images to a maximum size
    \adjustboxset{max size={0.9\linewidth}{0.9\paperheight}}
    \usepackage{float}
    \floatplacement{figure}{H} % forces figures to be placed at the correct location
    \usepackage{xcolor} % Allow colors to be defined
    \usepackage{enumerate} % Needed for markdown enumerations to work
    \usepackage{geometry} % Used to adjust the document margins
    \usepackage{amsmath} % Equations
    \usepackage{amssymb} % Equations
    \usepackage{textcomp} % defines textquotesingle
    % Hack from http://tex.stackexchange.com/a/47451/13684:
    \AtBeginDocument{%
        \def\PYZsq{\textquotesingle}% Upright quotes in Pygmentized code
    }
    \usepackage{upquote} % Upright quotes for verbatim code
    \usepackage{eurosym} % defines \euro
    \usepackage[mathletters]{ucs} % Extended unicode (utf-8) support
    \usepackage{fancyvrb} % verbatim replacement that allows latex
    \usepackage{grffile} % extends the file name processing of package graphics 
                         % to support a larger range
    \makeatletter % fix for grffile with XeLaTeX
    \def\Gread@@xetex#1{%
      \IfFileExists{"\Gin@base".bb}%
      {\Gread@eps{\Gin@base.bb}}%
      {\Gread@@xetex@aux#1}%
    }
    \makeatother

    % The hyperref package gives us a pdf with properly built
    % internal navigation ('pdf bookmarks' for the table of contents,
    % internal cross-reference links, web links for URLs, etc.)
    \usepackage{hyperref}
    % The default LaTeX title has an obnoxious amount of whitespace. By default,
    % titling removes some of it. It also provides customization options.
    \usepackage{titling}
    \usepackage{longtable} % longtable support required by pandoc >1.10
    \usepackage{booktabs}  % table support for pandoc > 1.12.2
    \usepackage[inline]{enumitem} % IRkernel/repr support (it uses the enumerate* environment)
    \usepackage[normalem]{ulem} % ulem is needed to support strikethroughs (\sout)
                                % normalem makes italics be italics, not underlines
    \usepackage{mathrsfs}
    

    
    % Colors for the hyperref package
    \definecolor{urlcolor}{rgb}{0,.145,.698}
    \definecolor{linkcolor}{rgb}{.71,0.21,0.01}
    \definecolor{citecolor}{rgb}{.12,.54,.11}

    % ANSI colors
    \definecolor{ansi-black}{HTML}{3E424D}
    \definecolor{ansi-black-intense}{HTML}{282C36}
    \definecolor{ansi-red}{HTML}{E75C58}
    \definecolor{ansi-red-intense}{HTML}{B22B31}
    \definecolor{ansi-green}{HTML}{00A250}
    \definecolor{ansi-green-intense}{HTML}{007427}
    \definecolor{ansi-yellow}{HTML}{DDB62B}
    \definecolor{ansi-yellow-intense}{HTML}{B27D12}
    \definecolor{ansi-blue}{HTML}{208FFB}
    \definecolor{ansi-blue-intense}{HTML}{0065CA}
    \definecolor{ansi-magenta}{HTML}{D160C4}
    \definecolor{ansi-magenta-intense}{HTML}{A03196}
    \definecolor{ansi-cyan}{HTML}{60C6C8}
    \definecolor{ansi-cyan-intense}{HTML}{258F8F}
    \definecolor{ansi-white}{HTML}{C5C1B4}
    \definecolor{ansi-white-intense}{HTML}{A1A6B2}
    \definecolor{ansi-default-inverse-fg}{HTML}{FFFFFF}
    \definecolor{ansi-default-inverse-bg}{HTML}{000000}

    % commands and environments needed by pandoc snippets
    % extracted from the output of `pandoc -s`
    \providecommand{\tightlist}{%
      \setlength{\itemsep}{0pt}\setlength{\parskip}{0pt}}
    \DefineVerbatimEnvironment{Highlighting}{Verbatim}{commandchars=\\\{\}}
    % Add ',fontsize=\small' for more characters per line
    \newenvironment{Shaded}{}{}
    \newcommand{\KeywordTok}[1]{\textcolor[rgb]{0.00,0.44,0.13}{\textbf{{#1}}}}
    \newcommand{\DataTypeTok}[1]{\textcolor[rgb]{0.56,0.13,0.00}{{#1}}}
    \newcommand{\DecValTok}[1]{\textcolor[rgb]{0.25,0.63,0.44}{{#1}}}
    \newcommand{\BaseNTok}[1]{\textcolor[rgb]{0.25,0.63,0.44}{{#1}}}
    \newcommand{\FloatTok}[1]{\textcolor[rgb]{0.25,0.63,0.44}{{#1}}}
    \newcommand{\CharTok}[1]{\textcolor[rgb]{0.25,0.44,0.63}{{#1}}}
    \newcommand{\StringTok}[1]{\textcolor[rgb]{0.25,0.44,0.63}{{#1}}}
    \newcommand{\CommentTok}[1]{\textcolor[rgb]{0.38,0.63,0.69}{\textit{{#1}}}}
    \newcommand{\OtherTok}[1]{\textcolor[rgb]{0.00,0.44,0.13}{{#1}}}
    \newcommand{\AlertTok}[1]{\textcolor[rgb]{1.00,0.00,0.00}{\textbf{{#1}}}}
    \newcommand{\FunctionTok}[1]{\textcolor[rgb]{0.02,0.16,0.49}{{#1}}}
    \newcommand{\RegionMarkerTok}[1]{{#1}}
    \newcommand{\ErrorTok}[1]{\textcolor[rgb]{1.00,0.00,0.00}{\textbf{{#1}}}}
    \newcommand{\NormalTok}[1]{{#1}}
    
    % Additional commands for more recent versions of Pandoc
    \newcommand{\ConstantTok}[1]{\textcolor[rgb]{0.53,0.00,0.00}{{#1}}}
    \newcommand{\SpecialCharTok}[1]{\textcolor[rgb]{0.25,0.44,0.63}{{#1}}}
    \newcommand{\VerbatimStringTok}[1]{\textcolor[rgb]{0.25,0.44,0.63}{{#1}}}
    \newcommand{\SpecialStringTok}[1]{\textcolor[rgb]{0.73,0.40,0.53}{{#1}}}
    \newcommand{\ImportTok}[1]{{#1}}
    \newcommand{\DocumentationTok}[1]{\textcolor[rgb]{0.73,0.13,0.13}{\textit{{#1}}}}
    \newcommand{\AnnotationTok}[1]{\textcolor[rgb]{0.38,0.63,0.69}{\textbf{\textit{{#1}}}}}
    \newcommand{\CommentVarTok}[1]{\textcolor[rgb]{0.38,0.63,0.69}{\textbf{\textit{{#1}}}}}
    \newcommand{\VariableTok}[1]{\textcolor[rgb]{0.10,0.09,0.49}{{#1}}}
    \newcommand{\ControlFlowTok}[1]{\textcolor[rgb]{0.00,0.44,0.13}{\textbf{{#1}}}}
    \newcommand{\OperatorTok}[1]{\textcolor[rgb]{0.40,0.40,0.40}{{#1}}}
    \newcommand{\BuiltInTok}[1]{{#1}}
    \newcommand{\ExtensionTok}[1]{{#1}}
    \newcommand{\PreprocessorTok}[1]{\textcolor[rgb]{0.74,0.48,0.00}{{#1}}}
    \newcommand{\AttributeTok}[1]{\textcolor[rgb]{0.49,0.56,0.16}{{#1}}}
    \newcommand{\InformationTok}[1]{\textcolor[rgb]{0.38,0.63,0.69}{\textbf{\textit{{#1}}}}}
    \newcommand{\WarningTok}[1]{\textcolor[rgb]{0.38,0.63,0.69}{\textbf{\textit{{#1}}}}}
    
    
    % Define a nice break command that doesn't care if a line doesn't already
    % exist.
    \def\br{\hspace*{\fill} \\* }
    % Math Jax compatibility definitions
    \def\gt{>}
    \def\lt{<}
    \let\Oldtex\TeX
    \let\Oldlatex\LaTeX
    \renewcommand{\TeX}{\textrm{\Oldtex}}
    \renewcommand{\LaTeX}{\textrm{\Oldlatex}}
    % Document parameters
    % Document title
    \title{Avaliação 1 introdução a fisica quantica}
    
    
    
    
    
% Pygments definitions
\makeatletter
\def\PY@reset{\let\PY@it=\relax \let\PY@bf=\relax%
    \let\PY@ul=\relax \let\PY@tc=\relax%
    \let\PY@bc=\relax \let\PY@ff=\relax}
\def\PY@tok#1{\csname PY@tok@#1\endcsname}
\def\PY@toks#1+{\ifx\relax#1\empty\else%
    \PY@tok{#1}\expandafter\PY@toks\fi}
\def\PY@do#1{\PY@bc{\PY@tc{\PY@ul{%
    \PY@it{\PY@bf{\PY@ff{#1}}}}}}}
\def\PY#1#2{\PY@reset\PY@toks#1+\relax+\PY@do{#2}}

\expandafter\def\csname PY@tok@w\endcsname{\def\PY@tc##1{\textcolor[rgb]{0.73,0.73,0.73}{##1}}}
\expandafter\def\csname PY@tok@c\endcsname{\let\PY@it=\textit\def\PY@tc##1{\textcolor[rgb]{0.25,0.50,0.50}{##1}}}
\expandafter\def\csname PY@tok@cp\endcsname{\def\PY@tc##1{\textcolor[rgb]{0.74,0.48,0.00}{##1}}}
\expandafter\def\csname PY@tok@k\endcsname{\let\PY@bf=\textbf\def\PY@tc##1{\textcolor[rgb]{0.00,0.50,0.00}{##1}}}
\expandafter\def\csname PY@tok@kp\endcsname{\def\PY@tc##1{\textcolor[rgb]{0.00,0.50,0.00}{##1}}}
\expandafter\def\csname PY@tok@kt\endcsname{\def\PY@tc##1{\textcolor[rgb]{0.69,0.00,0.25}{##1}}}
\expandafter\def\csname PY@tok@o\endcsname{\def\PY@tc##1{\textcolor[rgb]{0.40,0.40,0.40}{##1}}}
\expandafter\def\csname PY@tok@ow\endcsname{\let\PY@bf=\textbf\def\PY@tc##1{\textcolor[rgb]{0.67,0.13,1.00}{##1}}}
\expandafter\def\csname PY@tok@nb\endcsname{\def\PY@tc##1{\textcolor[rgb]{0.00,0.50,0.00}{##1}}}
\expandafter\def\csname PY@tok@nf\endcsname{\def\PY@tc##1{\textcolor[rgb]{0.00,0.00,1.00}{##1}}}
\expandafter\def\csname PY@tok@nc\endcsname{\let\PY@bf=\textbf\def\PY@tc##1{\textcolor[rgb]{0.00,0.00,1.00}{##1}}}
\expandafter\def\csname PY@tok@nn\endcsname{\let\PY@bf=\textbf\def\PY@tc##1{\textcolor[rgb]{0.00,0.00,1.00}{##1}}}
\expandafter\def\csname PY@tok@ne\endcsname{\let\PY@bf=\textbf\def\PY@tc##1{\textcolor[rgb]{0.82,0.25,0.23}{##1}}}
\expandafter\def\csname PY@tok@nv\endcsname{\def\PY@tc##1{\textcolor[rgb]{0.10,0.09,0.49}{##1}}}
\expandafter\def\csname PY@tok@no\endcsname{\def\PY@tc##1{\textcolor[rgb]{0.53,0.00,0.00}{##1}}}
\expandafter\def\csname PY@tok@nl\endcsname{\def\PY@tc##1{\textcolor[rgb]{0.63,0.63,0.00}{##1}}}
\expandafter\def\csname PY@tok@ni\endcsname{\let\PY@bf=\textbf\def\PY@tc##1{\textcolor[rgb]{0.60,0.60,0.60}{##1}}}
\expandafter\def\csname PY@tok@na\endcsname{\def\PY@tc##1{\textcolor[rgb]{0.49,0.56,0.16}{##1}}}
\expandafter\def\csname PY@tok@nt\endcsname{\let\PY@bf=\textbf\def\PY@tc##1{\textcolor[rgb]{0.00,0.50,0.00}{##1}}}
\expandafter\def\csname PY@tok@nd\endcsname{\def\PY@tc##1{\textcolor[rgb]{0.67,0.13,1.00}{##1}}}
\expandafter\def\csname PY@tok@s\endcsname{\def\PY@tc##1{\textcolor[rgb]{0.73,0.13,0.13}{##1}}}
\expandafter\def\csname PY@tok@sd\endcsname{\let\PY@it=\textit\def\PY@tc##1{\textcolor[rgb]{0.73,0.13,0.13}{##1}}}
\expandafter\def\csname PY@tok@si\endcsname{\let\PY@bf=\textbf\def\PY@tc##1{\textcolor[rgb]{0.73,0.40,0.53}{##1}}}
\expandafter\def\csname PY@tok@se\endcsname{\let\PY@bf=\textbf\def\PY@tc##1{\textcolor[rgb]{0.73,0.40,0.13}{##1}}}
\expandafter\def\csname PY@tok@sr\endcsname{\def\PY@tc##1{\textcolor[rgb]{0.73,0.40,0.53}{##1}}}
\expandafter\def\csname PY@tok@ss\endcsname{\def\PY@tc##1{\textcolor[rgb]{0.10,0.09,0.49}{##1}}}
\expandafter\def\csname PY@tok@sx\endcsname{\def\PY@tc##1{\textcolor[rgb]{0.00,0.50,0.00}{##1}}}
\expandafter\def\csname PY@tok@m\endcsname{\def\PY@tc##1{\textcolor[rgb]{0.40,0.40,0.40}{##1}}}
\expandafter\def\csname PY@tok@gh\endcsname{\let\PY@bf=\textbf\def\PY@tc##1{\textcolor[rgb]{0.00,0.00,0.50}{##1}}}
\expandafter\def\csname PY@tok@gu\endcsname{\let\PY@bf=\textbf\def\PY@tc##1{\textcolor[rgb]{0.50,0.00,0.50}{##1}}}
\expandafter\def\csname PY@tok@gd\endcsname{\def\PY@tc##1{\textcolor[rgb]{0.63,0.00,0.00}{##1}}}
\expandafter\def\csname PY@tok@gi\endcsname{\def\PY@tc##1{\textcolor[rgb]{0.00,0.63,0.00}{##1}}}
\expandafter\def\csname PY@tok@gr\endcsname{\def\PY@tc##1{\textcolor[rgb]{1.00,0.00,0.00}{##1}}}
\expandafter\def\csname PY@tok@ge\endcsname{\let\PY@it=\textit}
\expandafter\def\csname PY@tok@gs\endcsname{\let\PY@bf=\textbf}
\expandafter\def\csname PY@tok@gp\endcsname{\let\PY@bf=\textbf\def\PY@tc##1{\textcolor[rgb]{0.00,0.00,0.50}{##1}}}
\expandafter\def\csname PY@tok@go\endcsname{\def\PY@tc##1{\textcolor[rgb]{0.53,0.53,0.53}{##1}}}
\expandafter\def\csname PY@tok@gt\endcsname{\def\PY@tc##1{\textcolor[rgb]{0.00,0.27,0.87}{##1}}}
\expandafter\def\csname PY@tok@err\endcsname{\def\PY@bc##1{\setlength{\fboxsep}{0pt}\fcolorbox[rgb]{1.00,0.00,0.00}{1,1,1}{\strut ##1}}}
\expandafter\def\csname PY@tok@kc\endcsname{\let\PY@bf=\textbf\def\PY@tc##1{\textcolor[rgb]{0.00,0.50,0.00}{##1}}}
\expandafter\def\csname PY@tok@kd\endcsname{\let\PY@bf=\textbf\def\PY@tc##1{\textcolor[rgb]{0.00,0.50,0.00}{##1}}}
\expandafter\def\csname PY@tok@kn\endcsname{\let\PY@bf=\textbf\def\PY@tc##1{\textcolor[rgb]{0.00,0.50,0.00}{##1}}}
\expandafter\def\csname PY@tok@kr\endcsname{\let\PY@bf=\textbf\def\PY@tc##1{\textcolor[rgb]{0.00,0.50,0.00}{##1}}}
\expandafter\def\csname PY@tok@bp\endcsname{\def\PY@tc##1{\textcolor[rgb]{0.00,0.50,0.00}{##1}}}
\expandafter\def\csname PY@tok@fm\endcsname{\def\PY@tc##1{\textcolor[rgb]{0.00,0.00,1.00}{##1}}}
\expandafter\def\csname PY@tok@vc\endcsname{\def\PY@tc##1{\textcolor[rgb]{0.10,0.09,0.49}{##1}}}
\expandafter\def\csname PY@tok@vg\endcsname{\def\PY@tc##1{\textcolor[rgb]{0.10,0.09,0.49}{##1}}}
\expandafter\def\csname PY@tok@vi\endcsname{\def\PY@tc##1{\textcolor[rgb]{0.10,0.09,0.49}{##1}}}
\expandafter\def\csname PY@tok@vm\endcsname{\def\PY@tc##1{\textcolor[rgb]{0.10,0.09,0.49}{##1}}}
\expandafter\def\csname PY@tok@sa\endcsname{\def\PY@tc##1{\textcolor[rgb]{0.73,0.13,0.13}{##1}}}
\expandafter\def\csname PY@tok@sb\endcsname{\def\PY@tc##1{\textcolor[rgb]{0.73,0.13,0.13}{##1}}}
\expandafter\def\csname PY@tok@sc\endcsname{\def\PY@tc##1{\textcolor[rgb]{0.73,0.13,0.13}{##1}}}
\expandafter\def\csname PY@tok@dl\endcsname{\def\PY@tc##1{\textcolor[rgb]{0.73,0.13,0.13}{##1}}}
\expandafter\def\csname PY@tok@s2\endcsname{\def\PY@tc##1{\textcolor[rgb]{0.73,0.13,0.13}{##1}}}
\expandafter\def\csname PY@tok@sh\endcsname{\def\PY@tc##1{\textcolor[rgb]{0.73,0.13,0.13}{##1}}}
\expandafter\def\csname PY@tok@s1\endcsname{\def\PY@tc##1{\textcolor[rgb]{0.73,0.13,0.13}{##1}}}
\expandafter\def\csname PY@tok@mb\endcsname{\def\PY@tc##1{\textcolor[rgb]{0.40,0.40,0.40}{##1}}}
\expandafter\def\csname PY@tok@mf\endcsname{\def\PY@tc##1{\textcolor[rgb]{0.40,0.40,0.40}{##1}}}
\expandafter\def\csname PY@tok@mh\endcsname{\def\PY@tc##1{\textcolor[rgb]{0.40,0.40,0.40}{##1}}}
\expandafter\def\csname PY@tok@mi\endcsname{\def\PY@tc##1{\textcolor[rgb]{0.40,0.40,0.40}{##1}}}
\expandafter\def\csname PY@tok@il\endcsname{\def\PY@tc##1{\textcolor[rgb]{0.40,0.40,0.40}{##1}}}
\expandafter\def\csname PY@tok@mo\endcsname{\def\PY@tc##1{\textcolor[rgb]{0.40,0.40,0.40}{##1}}}
\expandafter\def\csname PY@tok@ch\endcsname{\let\PY@it=\textit\def\PY@tc##1{\textcolor[rgb]{0.25,0.50,0.50}{##1}}}
\expandafter\def\csname PY@tok@cm\endcsname{\let\PY@it=\textit\def\PY@tc##1{\textcolor[rgb]{0.25,0.50,0.50}{##1}}}
\expandafter\def\csname PY@tok@cpf\endcsname{\let\PY@it=\textit\def\PY@tc##1{\textcolor[rgb]{0.25,0.50,0.50}{##1}}}
\expandafter\def\csname PY@tok@c1\endcsname{\let\PY@it=\textit\def\PY@tc##1{\textcolor[rgb]{0.25,0.50,0.50}{##1}}}
\expandafter\def\csname PY@tok@cs\endcsname{\let\PY@it=\textit\def\PY@tc##1{\textcolor[rgb]{0.25,0.50,0.50}{##1}}}

\def\PYZbs{\char`\\}
\def\PYZus{\char`\_}
\def\PYZob{\char`\{}
\def\PYZcb{\char`\}}
\def\PYZca{\char`\^}
\def\PYZam{\char`\&}
\def\PYZlt{\char`\<}
\def\PYZgt{\char`\>}
\def\PYZsh{\char`\#}
\def\PYZpc{\char`\%}
\def\PYZdl{\char`\$}
\def\PYZhy{\char`\-}
\def\PYZsq{\char`\'}
\def\PYZdq{\char`\"}
\def\PYZti{\char`\~}
% for compatibility with earlier versions
\def\PYZat{@}
\def\PYZlb{[}
\def\PYZrb{]}
\makeatother


    % For linebreaks inside Verbatim environment from package fancyvrb. 
    \makeatletter
        \newbox\Wrappedcontinuationbox 
        \newbox\Wrappedvisiblespacebox 
        \newcommand*\Wrappedvisiblespace {\textcolor{red}{\textvisiblespace}} 
        \newcommand*\Wrappedcontinuationsymbol {\textcolor{red}{\llap{\tiny$\m@th\hookrightarrow$}}} 
        \newcommand*\Wrappedcontinuationindent {3ex } 
        \newcommand*\Wrappedafterbreak {\kern\Wrappedcontinuationindent\copy\Wrappedcontinuationbox} 
        % Take advantage of the already applied Pygments mark-up to insert 
        % potential linebreaks for TeX processing. 
        %        {, <, #, %, $, ' and ": go to next line. 
        %        _, }, ^, &, >, - and ~: stay at end of broken line. 
        % Use of \textquotesingle for straight quote. 
        \newcommand*\Wrappedbreaksatspecials {% 
            \def\PYGZus{\discretionary{\char`\_}{\Wrappedafterbreak}{\char`\_}}% 
            \def\PYGZob{\discretionary{}{\Wrappedafterbreak\char`\{}{\char`\{}}% 
            \def\PYGZcb{\discretionary{\char`\}}{\Wrappedafterbreak}{\char`\}}}% 
            \def\PYGZca{\discretionary{\char`\^}{\Wrappedafterbreak}{\char`\^}}% 
            \def\PYGZam{\discretionary{\char`\&}{\Wrappedafterbreak}{\char`\&}}% 
            \def\PYGZlt{\discretionary{}{\Wrappedafterbreak\char`\<}{\char`\<}}% 
            \def\PYGZgt{\discretionary{\char`\>}{\Wrappedafterbreak}{\char`\>}}% 
            \def\PYGZsh{\discretionary{}{\Wrappedafterbreak\char`\#}{\char`\#}}% 
            \def\PYGZpc{\discretionary{}{\Wrappedafterbreak\char`\%}{\char`\%}}% 
            \def\PYGZdl{\discretionary{}{\Wrappedafterbreak\char`\$}{\char`\$}}% 
            \def\PYGZhy{\discretionary{\char`\-}{\Wrappedafterbreak}{\char`\-}}% 
            \def\PYGZsq{\discretionary{}{\Wrappedafterbreak\textquotesingle}{\textquotesingle}}% 
            \def\PYGZdq{\discretionary{}{\Wrappedafterbreak\char`\"}{\char`\"}}% 
            \def\PYGZti{\discretionary{\char`\~}{\Wrappedafterbreak}{\char`\~}}% 
        } 
        % Some characters . , ; ? ! / are not pygmentized. 
        % This macro makes them "active" and they will insert potential linebreaks 
        \newcommand*\Wrappedbreaksatpunct {% 
            \lccode`\~`\.\lowercase{\def~}{\discretionary{\hbox{\char`\.}}{\Wrappedafterbreak}{\hbox{\char`\.}}}% 
            \lccode`\~`\,\lowercase{\def~}{\discretionary{\hbox{\char`\,}}{\Wrappedafterbreak}{\hbox{\char`\,}}}% 
            \lccode`\~`\;\lowercase{\def~}{\discretionary{\hbox{\char`\;}}{\Wrappedafterbreak}{\hbox{\char`\;}}}% 
            \lccode`\~`\:\lowercase{\def~}{\discretionary{\hbox{\char`\:}}{\Wrappedafterbreak}{\hbox{\char`\:}}}% 
            \lccode`\~`\?\lowercase{\def~}{\discretionary{\hbox{\char`\?}}{\Wrappedafterbreak}{\hbox{\char`\?}}}% 
            \lccode`\~`\!\lowercase{\def~}{\discretionary{\hbox{\char`\!}}{\Wrappedafterbreak}{\hbox{\char`\!}}}% 
            \lccode`\~`\/\lowercase{\def~}{\discretionary{\hbox{\char`\/}}{\Wrappedafterbreak}{\hbox{\char`\/}}}% 
            \catcode`\.\active
            \catcode`\,\active 
            \catcode`\;\active
            \catcode`\:\active
            \catcode`\?\active
            \catcode`\!\active
            \catcode`\/\active 
            \lccode`\~`\~ 	
        }
    \makeatother

    \let\OriginalVerbatim=\Verbatim
    \makeatletter
    \renewcommand{\Verbatim}[1][1]{%
        %\parskip\z@skip
        \sbox\Wrappedcontinuationbox {\Wrappedcontinuationsymbol}%
        \sbox\Wrappedvisiblespacebox {\FV@SetupFont\Wrappedvisiblespace}%
        \def\FancyVerbFormatLine ##1{\hsize\linewidth
            \vtop{\raggedright\hyphenpenalty\z@\exhyphenpenalty\z@
                \doublehyphendemerits\z@\finalhyphendemerits\z@
                \strut ##1\strut}%
        }%
        % If the linebreak is at a space, the latter will be displayed as visible
        % space at end of first line, and a continuation symbol starts next line.
        % Stretch/shrink are however usually zero for typewriter font.
        \def\FV@Space {%
            \nobreak\hskip\z@ plus\fontdimen3\font minus\fontdimen4\font
            \discretionary{\copy\Wrappedvisiblespacebox}{\Wrappedafterbreak}
            {\kern\fontdimen2\font}%
        }%
        
        % Allow breaks at special characters using \PYG... macros.
        \Wrappedbreaksatspecials
        % Breaks at punctuation characters . , ; ? ! and / need catcode=\active 	
        \OriginalVerbatim[#1,codes*=\Wrappedbreaksatpunct]%
    }
    \makeatother

    % Exact colors from NB
    \definecolor{incolor}{HTML}{303F9F}
    \definecolor{outcolor}{HTML}{D84315}
    \definecolor{cellborder}{HTML}{CFCFCF}
    \definecolor{cellbackground}{HTML}{F7F7F7}
    
    % prompt
    \makeatletter
    \newcommand{\boxspacing}{\kern\kvtcb@left@rule\kern\kvtcb@boxsep}
    \makeatother
    \newcommand{\prompt}[4]{
        \ttfamily\llap{{\color{#2}[#3]:\hspace{3pt}#4}}\vspace{-\baselineskip}
    }
    

    
    % Prevent overflowing lines due to hard-to-break entities
    \sloppy 
    % Setup hyperref package
    \hypersetup{
      breaklinks=true,  % so long urls are correctly broken across lines
      colorlinks=true,
      urlcolor=urlcolor,
      linkcolor=linkcolor,
      citecolor=citecolor,
      }
    % Slightly bigger margins than the latex defaults
    
    \geometry{verbose,tmargin=1in,bmargin=1in,lmargin=1in,rmargin=1in}
    
    

\begin{document}
    
    \maketitle
    
    

    
    \hypertarget{uma-dissertauxe7uxe3o-sobre-o-problema-da-radiauxe7uxe3o-tuxe9rmica.}{%
\section{Uma dissertação sobre o problema da radiação
térmica.}\label{uma-dissertauxe7uxe3o-sobre-o-problema-da-radiauxe7uxe3o-tuxe9rmica.}}

\hypertarget{cayro-neto}{%
\subsubsection{Cayro Neto}\label{cayro-neto}}

    \hypertarget{o-contexto}{%
\subsection{O Contexto}\label{o-contexto}}

O cenário da física no inicio do século XX era fantástico. Os físicos da
época estavam satisfeitos com toda a teoria classica sobre Mecânica e
eletromagnetismo. O modelo discreto de niveis de energia de um atomo
proposto por Boltzmann satisfazia as equações de Maxwell e havia um
grande consenso sobre a natureza da luz: é onda! A física estatística ja
havia avançado o suficiente para existir uma teoria muito concreta sobre
a relação de temperatura com a energia cinética de gases. Existiam
apenas dois problemas que seriam necéssarios para se ter todo o
conhecimento sobre a natureza: O espectro da radiação de um corpo negro
e o efeito fotoelétrico.

O século XIX foi muito importante para o desenvolvimento da
termodinâmica clássica. Muita atenção era dada para os estudos das
propriedades do calor em varios objetos. Nesse contexto, foi
desenvolvido um modelo de corpo negro ideal: um objeto que absorve toda
a radiação que o atinge e ao
\textgreater{}\textgreater{}\textgreater{}\textgreater{}mesmo tempo que
emite essa energia.\textless{}\textless{}\textless{}\textless{} Esse
modelo propunha que a energia irradiada poderia ser escrita como ondas
estacionárias dentro de uma cavidade de um corpo negro. A energia era
emitida na forma de ondas de frequência \(\nu\), e deveria ser
proporcional ao numero de modos de vibração dessa onda. E física
estatística da época tinha como estabelecido que a energia media de cada
modo é proporcional a kT. Mas os modos de vibração são proporcionais a
frequência emitida, e por isso era de se esperar que as energias fossem
cada vez mais altas a medida que as frequências fossem maiores. Isso não
é o que acontece na realidade, até porque se um corpo negro emite um
espectro continuo de frequências então esse modelo levaria a uma emissão
de ultravioleta que não condiz com a realidade.FALAR SOBRE EXPERIMENTO.

Até 1900, quando Plank apresentou uma solução para esse problema,
ninguém sabia explicar o porque do modelo téorico da época não condizia
com o observado. Acontece que esse problema foi um grande marco para
física quântica pois sua explicação leva a toda uma teoria que
revolucionou toda a maneira como a natureza é entendida pela ciência.

    \hypertarget{a-catastrofe}{%
\subsection{A Catastrofe}\label{a-catastrofe}}

Um dos resultados mais brilhantes das equações de maxwell é a previsão
de ondas eletromagnéticas. Usando apenas as equações é possivel deduzir
que a variação no campo elétrico de uma onda é dada por: . \[
\frac{\partial^2 E}{\partial x^2}+\frac{\partial^2 E}{\partial y^2}+\frac{\partial^2 E}{\partial z^2} =\frac{1}{c^2} \frac{\partial^2 s}{\partial t^2}
\]

Para o modelo em que um corpo negro é descrito como uma cavidade a
solução da equação de onda deve ter amplitude 0 nas paredes da cavidade,
até porque um valor diferente de zero implicaria na dissipação de
energia, o que vai contra a suposição que o sistema esta em equilíbrio.
Para formar uma onda estacionária, o caminho de reflexão ao redor da
cavidade deve produzir um caminho fechado. Através dessas condições de
contorno podemos escrever uma solução na forma:

\[
E=E_o sen \frac{n_1 \pi x}{L}  sen \frac{n_2 \pi y}{L}  sen \frac{n_3 \pi z}{L}  sen \frac{2 \pi c t}{\lambda} 
\]

E ao comparar essa solução com a equação de onda temos que:

\[
\begin{equation}
n_1^2+ n_2^2+ n_3^2=\frac{4L^2}{\lambda^2}
\label{n123} \tag{1}
\end{equation}
\]

Como descrito na seção anterior desse texto, é importente avaliar o
número de modos possiveis para essa condição, isto é, contar todas as
combinações possiveis para valores inteiros de n. Uma aproximação pode
ser feita tratando o numero de combinações possiveis como o volume
tridimencional de valores de n. Ao usar a reção de volume para uma
esfera, com os valores de n como 3 eixos diferentes de um espaço
tridimensional, é possivel escrever que:

\[
n=\frac{4 \pi}{3} (n_1^2+n_2^2+n_3^2)^{\frac{3}{2}}
\]

Porém, é necessario fazer algumas correções a respeito desse valor.
Primeiro que um espaço tridimensional admite valores negativos para
\(n_1\), \(n_2\) e \(n_3\), o que na realidade não acontece e por isso
consideramos apenas o espaço em que todos os n são positivos
(\(\frac{1}{8}\) do espaço total). Além disso, a onda pode estar
polarizada em duas direções perpendiculares, e por isso o valor deve ser
dobrado. O valor de n real fica:

\[
n=\frac{\pi}{3} (n_1^2+n_2^2+n_3^2)^{\frac{3}{2}}
\]

E de acordo com a equação \ref{n123}:

\[
n=\frac{8 \pi L^3}{3 \lambda^3}
\]

Tendo desenvolvida uma expressão para o numero de ondas estacionarias em
uma cavidade, é interessante conhecer a distribuição de numero de modos
de acordo com o comprimento de onda. Para isso basta que: \[
\begin{equation}
\frac{dn}{d\lambda}=\frac{d}{d\lambda}\frac{8 \pi L^3}{3 \lambda^3} = -\frac{8 \pi L^3}{ \lambda^4}
\label{dist} \tag{2}
\end{equation}
\]

E para obter a distribuição por unidade de volume é suficiente apenas
dividir a equação \ref{dist} pelo volume da cavidade (a qual
consideraremos cúbica, mas que no final não faz diferença): \[
\frac{1}{L^3}(-\frac{8 \pi L^3}{ \lambda^4})=-\frac{1}{L^3}\frac{dN}{d\lambda}=\frac{8\pi}{\lambda^4}
\]

Aqui, é necessario introduzir o teorema da equipartição. Em PESQUISAR
ANO DO TEOREMA DA EQUIPARTIÇÂO foi desenvolvido um teorema que relaciona
a temperatura de um sistema com a energia media de um modo de vibração
atraves da constante de Boltzmann. Como a cavidade possui n modos de
vibração, \(\langle E_{TOTAL} \rangle=nkT\). Assim, colocando u como a
densidade de energia:

\[
\frac{du}{d\lambda}=\frac{1}{L^3}\frac{dE}{d\lambda}=-kT\frac{1}{L^3}\frac{dN}{d\lambda} = \frac{8 \pi kT}{\lambda^4}
\]

Essa expressão tambem pode ser escrita em função da frequência apenas
usando a regra da cadeia: \[
\frac{du}{d\nu}=\frac{du}{d\lambda}\frac{d\lambda}{d \nu}=\frac{du}{d\lambda}\frac{d(c/\nu)}{d \nu}=\frac{du}{d\lambda}\frac{c}{\nu^2}=\frac{8 \pi kT\nu^4}{c^4}\frac{c}{\nu^2}=kT\frac{8 \pi \nu^2}{c^3}
\]

Se considerar que a energia irradiada perpendicular a um pequeno
incremento de area, é possivel perceber que metade da densidade de
energia está indo contra as paredes enquanto metade está saindo do
sistema em queilibrio termico. É interessante observar a potência a
partir de um ponto fora da cavidade. Se esse ponto está a uma distancia
x da da area de emissão, A potência irradiada por unidade de comprimento
de onda é igual a metade da densidade de energia por comprimento de onda
vezes a razão entre o volume e o tempo para a energia irradiada chegar
ao ponto x. Assim: \[
R(\lambda)=\frac{1}{2}\frac{du}{d\lambda}\frac{Ax}{t}=\frac{1}{2}\frac{du}{d\lambda}\frac{Act}{t}=\frac{du}{d\lambda}\frac{Ac}{2}
\] Fazendo uma correção angular em relação a area efetiva em um angulo
\(\theta\): \[
R(\lambda)=\frac{du}{d\lambda}\frac{Ac}{2}cos^2\theta
\] Fazendo a media sobre todos os angulos: \[
R(\lambda)=\frac {\frac{du}{d\lambda}\frac{Ac}{2}\langle cos^2\theta \rangle}{Area}=\frac{du}{d\lambda} \frac{c}{4}=\frac{8 \pi kT}{\lambda^4} \frac{c}{4} = \frac{2 \pi c k T}{\lambda^4}
\]

Essa equação é a equação de Rayleigh-Jeans. Como é possivel perceber
pela dedução, ela deriva do eletromagnetismo clássico e algumas
propriedades matématicas. É possivel perceber também que quanto menor o
comprimento de onda maior a radiancia, e quando o comprimento de onda
tende a zero a radiancia tente ao infinito! Com um código simples em
python é possivel observar o crescimento absurdo da radiancia a medida
que o comprimento de onda fica menor.

    \begin{tcolorbox}[breakable, size=fbox, boxrule=1pt, pad at break*=1mm,colback=cellbackground, colframe=cellborder]
\prompt{In}{incolor}{33}{\boxspacing}
\begin{Verbatim}[commandchars=\\\{\}]
\PY{k+kn}{import} \PY{n+nn}{numpy} \PY{k}{as} \PY{n+nn}{np}
\PY{k+kn}{import} \PY{n+nn}{matplotlib}\PY{n+nn}{.}\PY{n+nn}{pyplot} \PY{k}{as} \PY{n+nn}{plt}
\PY{k+kn}{from} \PY{n+nn}{matplotlib} \PY{k+kn}{import} \PY{n}{cm}
\PY{k+kn}{import} \PY{n+nn}{matplotlib}\PY{n+nn}{.}\PY{n+nn}{colors}
\PY{n}{cmap} \PY{o}{=} \PY{n}{plt}\PY{o}{.}\PY{n}{cm}\PY{o}{.}\PY{n}{rainbow}
\PY{o}{\PYZpc{}}\PY{k}{matplotlib} inline
\end{Verbatim}
\end{tcolorbox}

    \begin{tcolorbox}[breakable, size=fbox, boxrule=1pt, pad at break*=1mm,colback=cellbackground, colframe=cellborder]
\prompt{In}{incolor}{51}{\boxspacing}
\begin{Verbatim}[commandchars=\\\{\}]
\PY{n}{c}\PY{o}{=}\PY{l+m+mi}{3}\PY{o}{*}\PY{l+m+mi}{10}\PY{o}{\PYZca{}}\PY{l+m+mi}{8}
\PY{n}{k}\PY{o}{=}\PY{l+m+mf}{1.38064852}\PY{o}{*}\PY{l+m+mi}{10}\PY{o}{*}\PY{o}{*}\PY{o}{\PYZhy{}}\PY{l+m+mi}{23}
\PY{k}{def} \PY{n+nf}{R}\PY{p}{(}\PY{n}{l}\PY{p}{,}\PY{n}{T}\PY{p}{)}\PY{p}{:}
    \PY{k}{return} \PY{l+m+mi}{2}\PY{o}{*}\PY{n}{np}\PY{o}{.}\PY{n}{pi}\PY{o}{*}\PY{n}{c}\PY{o}{*}\PY{n}{T}\PY{o}{/}\PY{n}{l}\PY{o}{*}\PY{o}{*}\PY{l+m+mi}{4}

\PY{n}{L}\PY{o}{=}\PY{p}{[}\PY{l+m+mi}{10}\PY{o}{*}\PY{o}{*}\PY{n}{i} \PY{k}{for} \PY{n}{i} \PY{o+ow}{in} \PY{n}{np}\PY{o}{.}\PY{n}{linspace}\PY{p}{(}\PY{o}{\PYZhy{}}\PY{l+m+mi}{5}\PY{p}{,}\PY{l+m+mi}{5}\PY{p}{,}\PY{l+m+mi}{100}\PY{p}{)}\PY{p}{]}
\PY{n}{T}\PY{o}{=}\PY{p}{[}\PY{l+m+mi}{200}\PY{o}{*}\PY{n}{i} \PY{k}{for} \PY{n}{i} \PY{o+ow}{in} \PY{n+nb}{range}\PY{p}{(}\PY{l+m+mi}{1}\PY{p}{,}\PY{l+m+mi}{7}\PY{p}{)}\PY{p}{]}
\PY{n}{Rad}\PY{o}{=}\PY{p}{[}\PY{p}{]}
\PY{n}{C}\PY{o}{=}\PY{p}{[}\PY{p}{]}
\PY{k}{for} \PY{n}{t} \PY{o+ow}{in} \PY{n}{T}\PY{p}{:}
    \PY{n}{Rt}\PY{o}{=}\PY{p}{[}\PY{p}{]}
    \PY{n}{ct}\PY{o}{=}\PY{p}{[}\PY{p}{]}
    \PY{k}{for} \PY{n}{l} \PY{o+ow}{in} \PY{n}{L}\PY{p}{:}
        \PY{n}{Rt}\PY{o}{.}\PY{n}{append}\PY{p}{(}\PY{n}{np}\PY{o}{.}\PY{n}{log}\PY{p}{(}\PY{n}{R}\PY{p}{(}\PY{n}{l}\PY{p}{,}\PY{n}{t}\PY{p}{)}\PY{p}{)}\PY{o}{/}\PY{n}{np}\PY{o}{.}\PY{n}{log}\PY{p}{(}\PY{l+m+mi}{10}\PY{p}{)}\PY{p}{)}
        \PY{n}{ct}\PY{o}{.}\PY{n}{append}\PY{p}{(}\PY{n}{t}\PY{o}{*}\PY{l+m+mi}{50}\PY{p}{)}
    \PY{n}{Rad}\PY{o}{.}\PY{n}{append}\PY{p}{(}\PY{n}{Rt}\PY{p}{)}
    \PY{n}{C}\PY{o}{.}\PY{n}{append}\PY{p}{(}\PY{n}{ct}\PY{p}{)}

\PY{k}{for} \PY{n}{i} \PY{o+ow}{in} \PY{n+nb}{range}\PY{p}{(}\PY{n+nb}{len}\PY{p}{(}\PY{n}{T}\PY{p}{)}\PY{p}{)}\PY{p}{:}
    \PY{n}{plt}\PY{o}{.}\PY{n}{plot}\PY{p}{(}\PY{n}{L}\PY{p}{,}\PY{n}{Rad}\PY{p}{[}\PY{n}{i}\PY{p}{]}\PY{p}{,} \PY{n}{label}\PY{o}{=}\PY{l+s+s1}{\PYZsq{}}\PY{l+s+s1}{T=}\PY{l+s+s1}{\PYZsq{}}\PY{o}{+}\PY{n+nb}{str}\PY{p}{(}\PY{n}{T}\PY{p}{[}\PY{n}{i}\PY{p}{]}\PY{p}{)}\PY{p}{)}
\PY{n}{plt}\PY{o}{.}\PY{n}{legend}\PY{p}{(}\PY{p}{)}
\end{Verbatim}
\end{tcolorbox}

            \begin{tcolorbox}[breakable, size=fbox, boxrule=.5pt, pad at break*=1mm, opacityfill=0]
\prompt{Out}{outcolor}{51}{\boxspacing}
\begin{Verbatim}[commandchars=\\\{\}]
<matplotlib.legend.Legend at 0x266fdf103c8>
\end{Verbatim}
\end{tcolorbox}
        
    \begin{center}
    \adjustimage{max size={0.9\linewidth}{0.9\paperheight}}{output_4_1.png}
    \end{center}
    { \hspace*{\fill} \\}
    
    Como é possivel ver no grafico anterior, para valores muito pequenos de
comprimento de onda a radiancia estrapola para valores absurdos!

    \hypertarget{a-soluuxe7uxe3o}{%
\subsection{A Solução}\label{a-soluuxe7uxe3o}}

A solução para o problema do corpo negro foi proposta por Plank numa
reunião da sociedade alemã de física no final de 1900. Com uma proposta
radical e sem uma explicação concreta de sua motivação, Plank propôs que
a energia emitida e absorvida por um corpo negro poderia ser quantizada,
como em pequenos pacotes de energia -----(os quantum). Ele propôs que a
energia deveria desses pacotes deveria ser proporcional a frequência da
onda sendo emitida. Essa hipótese também implica que para uma certa
temperatura existe uma intensidade máxima de radiação emitida pelo corpo
negro.

A medida que a frequência eletromagnetica da radiação cresce, a
magnitude associada com a energia também cresce. Ao assumir que a
energia pode ser emitida apenas em multiplos da frequência, Plank
deduziu uma euquação que encaixava muito bem com a curva experimental de
experimentos de corpo negro. A lógica é a seguinte: em baixas
temperaturas, a radiação é emitida apenas com frequencias baixas, o que
corresponde a pacotes de baixa energia. A medida que a temperatura do
objeto cresce, existe também um crescimento na probabilidade do corpo
negro emitir um pacote de alta energia (i.e.~um pacote de energia de
alta frequência). E em qualquer tempratura é mais provavel um corpo
negro emitir uma quantidade maior de pacotes de baixa energia do que uma
quantidade menor de pacotes de alta energia. Nesse sentido, existe
também uma frequência mais provavel de emissão, que é o
\{\{\{\{\{pico\}\}\}\}\}\} na distribuição espectral. Esse pico tende a
aumentar de frequência a medida que a temperatura aumenta.

A hipotese de plank explicava a distribuição expectral como uma
consequência da oscilação de eletrons. Max Plank se concentrou em criar
um modelo para cargas oscilantes que deveria existir nas paredes da
cavidade de um corpo negro, descrito na seção anterior. Assim, os
eletrons rradiam calor para dentro e - em equilíbrio termodinâmico -
emitem radiação para fora. Nesse sentido, os eletrons não irradiariam
energia de forma continua como era o esperado pela teoria clássica, mas
em intervalos quantizados de energia (chamados de quanta). Esses
intervalos de energia possuiam a mesma magnitude dos pequenos pacotes de
energia que Plank propos para a frequência emitida pelo corpo negro,
\(E=h\nu\).

A parte matemática é um pouco mais complexa que a da seção anterior.
Porém, é possivel deduzir a expressão de Plank para densidade de energia
a partir da lei de distribuição de boltzmann:

\[
P(E)=\frac{e^{\frac{-E}{kT}}}{kT}
\]

O teorema da equipartição citado anteriormente e os intervalos de
energia discretos \(\Delta E=h\nu\). Plank descreveu a densidade de
energia na cavidade como:

\[
\rho(\nu)==\frac{8 h \pi \nu^3}{c^3}\frac{1}{e^{\frac{h\nu}{kT}}-1}
\]

Que expresso em função do comprimento de onda fica:

\[
\rho(\lambda)=\frac{8 h \pi}{\lambda^5}\frac{1}{e^{\frac{hc}{\lambda kT}}-1}
\]

Usando mais um código em python é possivel observar a coerencia da Lei
de Plank para radiância espectral.

    \begin{tcolorbox}[breakable, size=fbox, boxrule=1pt, pad at break*=1mm,colback=cellbackground, colframe=cellborder]
\prompt{In}{incolor}{92}{\boxspacing}
\begin{Verbatim}[commandchars=\\\{\}]
\PY{n}{h}\PY{o}{=}\PY{l+m+mf}{6.63}\PY{o}{*}\PY{l+m+mi}{10}\PY{o}{*}\PY{o}{*}\PY{o}{\PYZhy{}}\PY{l+m+mi}{34}

\PY{k}{def} \PY{n+nf}{rho}\PY{p}{(}\PY{n}{wav}\PY{p}{,} \PY{n}{T}\PY{p}{)}\PY{p}{:}
    \PY{n}{a} \PY{o}{=} \PY{l+m+mf}{2.0}\PY{o}{*}\PY{n}{h}\PY{o}{*}\PY{n}{c}\PY{o}{*}\PY{o}{*}\PY{l+m+mi}{2}
    \PY{n}{b} \PY{o}{=} \PY{n}{h}\PY{o}{*}\PY{n}{c}\PY{o}{/}\PY{p}{(}\PY{n}{wav}\PY{o}{*}\PY{n}{k}\PY{o}{*}\PY{n}{T}\PY{p}{)}
    \PY{n}{I} \PY{o}{=} \PY{n}{a}\PY{o}{/} \PY{p}{(} \PY{p}{(}\PY{n}{wav}\PY{o}{*}\PY{o}{*}\PY{l+m+mi}{5}\PY{p}{)} \PY{o}{*} \PY{p}{(}\PY{n}{np}\PY{o}{.}\PY{n}{exp}\PY{p}{(}\PY{n}{b}\PY{p}{)} \PY{o}{\PYZhy{}} \PY{l+m+mf}{1.0}\PY{p}{)} \PY{p}{)}
    \PY{k}{return} \PY{n}{I}

\PY{n}{L}\PY{o}{=}\PY{n}{np}\PY{o}{.}\PY{n}{arange}\PY{p}{(}\PY{l+m+mf}{1e\PYZhy{}9}\PY{p}{,} \PY{l+m+mf}{3e\PYZhy{}6}\PY{p}{,} \PY{l+m+mf}{1e\PYZhy{}9}\PY{p}{)}
\PY{n}{T}\PY{o}{=}\PY{p}{[}\PY{l+m+mi}{1000}\PY{o}{*}\PY{n}{i} \PY{k}{for} \PY{n}{i} \PY{o+ow}{in} \PY{n+nb}{range}\PY{p}{(}\PY{l+m+mi}{1}\PY{p}{,}\PY{l+m+mi}{7}\PY{p}{)}\PY{p}{]}
\PY{n}{Rad}\PY{o}{=}\PY{p}{[}\PY{p}{]}
\PY{k}{for} \PY{n}{Tt} \PY{o+ow}{in} \PY{n}{T}\PY{p}{:}
    \PY{n}{Rad}\PY{o}{.}\PY{n}{append}\PY{p}{(}\PY{n}{rho}\PY{p}{(}\PY{n}{L}\PY{p}{,}\PY{n}{Tt}\PY{p}{)}\PY{p}{)}


\PY{k}{for} \PY{n}{i} \PY{o+ow}{in} \PY{n+nb}{range}\PY{p}{(}\PY{n+nb}{len}\PY{p}{(}\PY{n}{T}\PY{p}{)}\PY{p}{)}\PY{p}{:}
    \PY{n}{plt}\PY{o}{.}\PY{n}{plot}\PY{p}{(}\PY{n}{L}\PY{o}{*}\PY{l+m+mf}{1e9}\PY{p}{,}\PY{n}{Rad}\PY{p}{[}\PY{n}{i}\PY{p}{]}\PY{p}{,} \PY{n}{label}\PY{o}{=}\PY{l+s+s1}{\PYZsq{}}\PY{l+s+s1}{T=}\PY{l+s+s1}{\PYZsq{}}\PY{o}{+}\PY{n+nb}{str}\PY{p}{(}\PY{n}{T}\PY{p}{[}\PY{n}{i}\PY{p}{]}\PY{p}{)}\PY{o}{+}\PY{l+s+s1}{\PYZsq{}}\PY{l+s+s1}{K}\PY{l+s+s1}{\PYZsq{}}\PY{p}{)}

\PY{n}{plt}\PY{o}{.}\PY{n}{legend}\PY{p}{(}\PY{p}{)}
\end{Verbatim}
\end{tcolorbox}

    \begin{Verbatim}[commandchars=\\\{\}]
C:\textbackslash{}Users\textbackslash{}cayra\textbackslash{}anaconda3\textbackslash{}lib\textbackslash{}site-packages\textbackslash{}ipykernel\_launcher.py:6:
RuntimeWarning: overflow encountered in exp

    \end{Verbatim}

            \begin{tcolorbox}[breakable, size=fbox, boxrule=.5pt, pad at break*=1mm, opacityfill=0]
\prompt{Out}{outcolor}{92}{\boxspacing}
\begin{Verbatim}[commandchars=\\\{\}]
<matplotlib.legend.Legend at 0x2668352bc08>
\end{Verbatim}
\end{tcolorbox}
        
    \begin{center}
    \adjustimage{max size={0.9\linewidth}{0.9\paperheight}}{output_7_2.png}
    \end{center}
    { \hspace*{\fill} \\}
    
    \hypertarget{implicauxe7uxf5es}{%
\subsection{Implicações}\label{implicauxe7uxf5es}}

    \begin{tcolorbox}[breakable, size=fbox, boxrule=1pt, pad at break*=1mm,colback=cellbackground, colframe=cellborder]
\prompt{In}{incolor}{ }{\boxspacing}
\begin{Verbatim}[commandchars=\\\{\}]

\end{Verbatim}
\end{tcolorbox}


    % Add a bibliography block to the postdoc
    
    
    
\end{document}
